Governments worldwide are implementing policies to mitigate carbon emissions, necessitating significant economic resource reallocation. This study investigates the economic consequences of these shifts, focusing on quantifying reallocation costs and delineating conditions for optimal resource distribution. Utilizing a model of a closed economy characterized by monopolistic competition and firm heterogeneity, the study distinguishes between 'green' (eco-friendly) and 'brown' (fossil fuel-dependent) capital in the production function. The emission function has been extended to incorporate variable parameters that reflect the environmental impacts of these resources. This analysis aims to provide critical insights into the dynamics of resource allocation under environmental policy constraints, underscoring the trade-offs between economic output and environmental sustainability. The results are intended to guide policymaking by clarifying the economic costs associated with transitioning to a low-carbon economy and outlining strategies to balance environmental and economic objectives optimally.