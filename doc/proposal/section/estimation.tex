To calculate the parameter $\gamma_s$, indicative of the elasticity of substitution between capital and labor within sectors, a modified approach derived from the methodology presented in \cite{kmenta1967estimation} is employed. This method is specifically designed for estimating CES production functions, avoiding the direct use of structural assumptions from the model described in Section~\ref{sec:Model}. This choice is made to circumvent the theoretical scenario where an absence of financial restrictions, and hence no wedges, is assumed. Instead, this estimation method capitalizes on a different aspect of data variability than that used for determining wedge estimates, focusing on sector-level variations in the ratios of inputs to productivity.

In a hypothetical frictionless environment, one would not expect to find any correlation between these ratios and productivity levels, a situation that, according to \cite{kmenta1967estimation}, would imply an infinite elasticity of substitution. However, the presence of financial or environmental frictions introduces a tangible correlation between these variables, signifying a finite elasticity. Thus, the estimation of $\gamma_s$ serves to quantify the extent to which such frictions are uniformly present across firms within a sector, offering insights into average sector-specific variations in input ratios and productivity levels without pinpointing individual firm-level frictions.

As mentioned in Appendix~\ref{Ap:estimation}, the methodology for estimating the elasticity parameter $\gamma_s$ involves an ordinary least squares (OLS) analysis that incorporates firm-specific fixed effects. This approach is based on a regression derived from the first-order approximation of the production function at $\gamma_s = 1$, following the guidelines set by \cite{kmenta1967estimation}. An important aspect of this model is the nature of the error term, which is attributed to the variable $A_{si}$, indicative of the firm-specific productivity levels that are inherently correlated with the inputs. The identification of our model, particularly with the inclusion of fixed effects, is contingent upon the critical assumption that $A_{si}$'s variation is ob the firm level, manifesting without temporal fluctuations. This assumption is important for the robustness and validity of the OLS analysis, ensuring that the estimated $\gamma_s$ reflects the intrinsic elasticity of substitution between capital and labor, devoid of confounding temporal effects.

Exploring an alternative identification approach hinges on the premise that $A_{si}$, the firm-specific productivity term, follows an autoregressive pattern. This consideration allows for the application of a refined estimation strategy that employs quasi-differencing and the use of lagged instruments. This methodology draws inspiration from techniques similar to those outlined in \cite{blundell2000gmm} and \cite{davis2014macroeconomic}, offering a nuanced approach to capturing the dynamic aspects of productivity within firms.\footnote{
This alternative approach aligns with established strategies for production function estimation as seen in studies like \cite{olley1992dynamics}, \cite{levinsohn2003estimating}, and \cite{ackerberg2015identification}, underscoring the diversity of econometric methods tailored to understanding productivity dynamics.
} The potential challenges associated with employing the Generalized Method of Moments (GMM) in small samples, coupled with the size limitations of my industry-specific dataset, necessitate a cautious application of this method. Consequently, while the Ordinary Least Squares (OLS) method is utilized as the primary estimation technique, the GMM approach is also considered to verify the robustness of the results. This dual-method analysis ensures a comprehensive evaluation of the elasticity parameter $\gamma_s$, reinforcing the reliability of the conclusions drawn from the primary OLS estimation by corroborating them through an additional, albeit exploratory, GMM framework.