Following the data compilation approach detailed by \cite{martinsson2024effect}, my dataset will be constructed. They created their dataset through the amalgamation of plant- and company-level registry details, covering aspects such as financial metrics, employee numbers, and sector categorizations, alongside CO2 emission records from 1990 to 2015. The Swedish Environmental Protection Agency (SEPA) furnished the CO2 emission data at both the plant and company levels, which includes emissions accounted for under the EU ETS. Registry information for both public and private Swedish firms was sourced from Upplysningscentralen (UC) for the period of 1990-1997 and Bisnode Serrano from 1998 to 2015.

The dataset I am creating will focus on company-level information, capturing data on the resources firms deploy, the outcomes they produce, and their environmental impact. This encompasses details such as capital and labor (as inputs); sales, and value addition (as outputs); and CO2 emissions (to gauge environmental impact). Additionally, the dataset will feature information on the industry sector, geographic location, and ownership structure of each company.

One challenge in constructing this dataset is the difficulty in distinguishing between green capital, which is environmentally beneficial, and brown capital, which has a detrimental environmental impact, due to the lack of direct indicators. To overcome this obstacle, I will use a proxy to identify green capital, with abatement costs serving as a viable option. This is based on the premise that abatement costs are indicative of investments in environmentally friendly capital.

Furthermore, integrating this dataset with bond market data could provide valuable insights, particularly in establishment of the relationship between green bonds and green capital. This linkage could be another proxy for green capital, as firms issuing green bonds are likely to have a higher proportion of green capital in their asset base. 