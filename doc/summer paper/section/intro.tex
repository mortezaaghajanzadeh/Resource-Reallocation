In the face of escalating global environmental challenges, policymakers around the world have ramped up efforts to mitigate pollution and greenhouse gas emissions, deploying an arsenal of policy tools. These tools range from punitive measures like taxes or outright bans on environmentally detrimental investments, to incentives such as subsidies for adopting green technologies. Such policy measures have fundamentally shifted the landscape of marginal costs for businesses, compelling them to reevaluate their investment strategies. This policy-induced shift has catalyzed a movement towards new equilibriums, characterized by lower emission levels and a heightened focus on sustainable practices. However, this transition is not without its economic ramifications, as it entails not only shifts in corporate investment strategies but also significant economic reallocation and potential inefficiencies. These developments underscore the importance of scrutinizing the efficiency and economic ramifications of environmental policies, prompting a reevaluation of how effectively resources and inputs are allocated in this new context.

Building on the seminal research by \cite{whited2021misallocation} and\cite{hsieh2009misallocation}, which explores the adverse effects of resource misallocation on the economy's overall efficiency, this paper extends the conversation to the environmental policy. Misallocation, in essence, refers to a scenario where resources are not optimally distributed among firms, leading to suboptimal production outcomes. In the context of environmental policy, such misallocation is seen as a strategic reconfiguration of resources among firms, spurred by policy-induced shifts in their marginal costs. This dynamic, while aimed at encouraging more sustainable practices, introduces complexities in assessing the real economic impact of environmental policies. It raises pertinent questions about the trade-offs between economic efficiency and environmental sustainability, making it a fertile ground for inquiry.

To navigate these complexities, my model introduces novel elements to the \cite{hsieh2009misallocation}, starting with a critical distinction between 'brown' and 'green' capital. 'Brown' capital, characterized by its reliance on fossil fuels and its significant environmental footprint, contrasts sharply with 'green' capital, which embodies renewable and environmentally benign resources. This distinction is more than mere semantics; it acknowledges the heterogeneous impact of different types of capital on production and the environment. By moving beyond the traditional, homogenized view of capital, the model illuminates the nuanced roles these capital forms play in economic productivity and environmental sustainability. Furthermore, the introduction of a pollution function represents a significant leap forward. This function quantitatively links the utilization of brown and green capitals in production processes to their respective environmental impacts, enabling a detailed analysis of how production decisions affect ecological outcomes. This analytical enhancement allows the model to capture the complex interplay between economic activities and environmental health, offering insights into how shifts towards greener capital can mitigate environmental damage while fostering economic resilience.

To empirically anchor this theoretical exploration, I employ comprehensive registry data from a wide array of Swedish corporations, encompassing both publicly listed and privately held entities. Sweden's forefront position in enacting and enforcing some of the most stringent environmental regulations globally offers a unique vantage point to examine the economic repercussions of such policies. By meticulously analyzing this data, I aim to quantify the model's parameters and assess the extent and nature of economic reallocation in response to environmental policies. This empirical strategy not only validates the theoretical model but also provides a concrete basis for evaluating the efficacy and economic impacts of environmental regulations. Ultimately, this research endeavors to illuminate the pathways through which environmental policies can be calibrated to balance economic efficiency with the imperative of environmental sustainability, thereby contributing to the ongoing dialogue on sustainable development.