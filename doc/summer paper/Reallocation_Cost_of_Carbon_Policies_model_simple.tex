\documentclass[12pt]{article} % Set the font size to 12pt
\usepackage[margin=2.5cm]{geometry} % Set the margin 

% Packages
\usepackage{amsmath} % For mathematical symbols and equations
\usepackage{graphicx} % For including images
\usepackage{lipsum} % For generating dummy text
\usepackage{mathtools} % For mathematical symbols and equations
\usepackage{amssymb} % For mathematical symbols and equations
\usepackage{hyperref} % For hyperlinks
\usepackage{natbib} % For bibliography
\usepackage{booktabs} % For tables
% Document information
\title{Resource Reallocation with Carbon Emission Policies}
\author{Seyyed Morteza Aghajanzadeh}
\date{\today}

\begin{document}

\maketitle

\begin{abstract}
    Governments worldwide are implementing policies to mitigate carbon emissions, necessitating significant economic resource reallocation. This study investigates the economic consequences of these shifts, focusing on quantifying reallocation costs and delineating conditions for optimal resource distribution. Utilizing a model of a closed economy characterized by monopolistic competition and firm heterogeneity, the study distinguishes between 'green' (eco-friendly) and 'brown' (fossil fuel-dependent) capital in the production function. The emission function has been extended to incorporate variable parameters that reflect the environmental impacts of these resources. This analysis aims to provide critical insights into the dynamics of resource allocation under environmental policy constraints, underscoring the trade-offs between economic output and environmental sustainability. The results are intended to guide policymaking by clarifying the economic costs associated with transitioning to a low-carbon economy and outlining strategies to balance environmental and economic objectives optimally.    
\end{abstract}

\section*{Environment and Technology}

\begin{itemize}
    \item Total real output is:
    \begin{equation}
    Y = \Pi_1^S Y_s^{\lambda_s}, \quad \text{where} \quad \sum^S \lambda_s = 1
\end{equation}

    \item The real output in each sector $s$ is:
    \begin{equation}
    Y_s = \left(
        \sum_{i=1}^I Y_{si}^{\frac{\sigma_s-1}{\sigma_s}}
    \right)^{\frac{\sigma_s}{\sigma_s-1}}
\end{equation}

    \item The real output for firms $i$ in sector $s$ is:
	\begin{equation}
    \label{eq:firm_output}
    Y_{si} = \hat{A}_{si}\hat{K}_{si}^{\beta_s} L_{si}^{1-\beta_s}
    \quad,\qquad\hat{K} = (
        \alpha_s G_{si}^{\frac{\gamma_s-1}{\gamma_s}} + (1-\alpha_s) B_{si}^{\frac{\gamma_s-1}{\gamma_s}}
    ) ^ {\frac{\gamma_s}{\gamma_s-1}}
\end{equation}
    
	\item The firm's emission is:
	\begin{equation}
    E_{si} = {\tilde{A}_{si}} B_{si}
\end{equation}

    \item The nominal profit for firms:
    \begin{equation}
    \pi_{si} = (1+\tau_{s}^p) P_{s} Y_{s} - \left(\left[
        (1+ \tau_{G_{s}}) r_{si}G_{si} + (1+ \tau_{B_{s}}) r_{si}B_{si} + (1+ \tau_{l_{s}}) w_{si}l_{si}
    \right] + {\tau_{E} E_{si}}\right)
\end{equation}
    
\end{itemize}
\section*{Optimal Allocation}
\begin{itemize}
    \item To maximize profit, the firm follows a two-step process. First, it determines the optimal combination of capital and labor. Then, it selects the appropriate price level.
    \begin{equation*}
        \max_{{G_{si},B_{si},L_{si}}}  \quad
            - 		Cost \quad \text{s.t.} \quad \quad \hat{A}_{si}\hat{K}_{si}^{\beta_s} L_{si}^{1-\beta_s} = \bar{Y}_{si}
    \end{equation*}

    \begin{equation}
        {\frac{G_{si}}{B_{si}} = {z_{si}^k}} {=  \left(
            \frac{\alpha_s}{1-\alpha_s} \frac{(1+{\tau_s^B})r_{si} + {\tau_s^E}\tilde{A}}{(1+{\tau_s^G})r_{si}}
        \right) ^{\frac{1}{\gamma_s}}}
    \end{equation}

    \begin{equation}
        {\frac{L_{si}}{\hat{K}_{si}} = {z_{si}^l}}
        { = \frac{1-\beta}{\beta}\frac{1}{\alpha_s}\left(
            \alpha_s + (1-\alpha_s) {{z_{si}^k}}^{1-\gamma_s}
        \right)^{\frac{1}{1-\gamma_s}} \frac{(1+{\tau_s^G})r_{si}}{(1+{\tau_s^W})w_{si}}}
    \end{equation}
  

    \item Let's define the use the optimal ratios ($z^k_{si} \equiv (\frac{G_{si}}{B_{si}})*$ and $z^l_{si} \equiv (\frac{L_{si}}{\hat{K}_{si}})^*$)
    
    \item Optimal capital level for given output is (see appendix \ref{Ap:optimal level for given output} for proof):
    \begin{equation*}
    G_{si}^* = \frac{\bar{Y}_{si}}{\hat{A}_{si}} \left(
        \alpha_s  + (1-\alpha_s){z^k_{si}}^{(1-\gamma_s )}
    \right)^{\frac{\gamma_s}{1-\gamma_s}} {z_{si}^l}^{(1-\beta_s)}
\end{equation*}
\begin{equation*}
    B_{si}^* = \frac{\bar{Y}_{si}}{\hat{A}_{si}} \left(\alpha_s {z^k_{si}}^{\frac{\gamma_s -1}{\gamma_s}} + (1-\alpha_s)
    \right)^{\frac{\gamma_s}{1-\gamma_s}} {z_{si}^l}^{(\beta_s-1)}
\end{equation*}
\begin{equation*}
    L_{si}^* = \frac{\bar{Y}_{si}}{\hat{A}_{si}}  {z_{si}^l}^{-\beta_s}
\end{equation*}

    \item The emission level for given output is (see appendix \ref{Ap:emission optimal level} for proof):
    \begin{equation}
        {E_{si} = \frac{\tilde{A}_{si}}{\hat{A}_{si}}\left(
            \alpha_s {{z_{si}^k}}^{\gamma_s - 1} + (1-\alpha_s)
        \right) ^ {\frac{\gamma_s}{1-\gamma_s}} {{z_{si}^l}}^{1 - \beta} \bar{Y}_{si} }
        {= {\psi_{si}}\bar{Y}_{si}}
    \end{equation}
    
    \item The cost of production is (see appendix \ref{Ap:cost minimization function} for details of the definition):
    \begin{equation*}
    \begin{split}
        \Rightarrow C(\bar{F}_{si}) &  = \left[
            (1+ \tau_{G_{si}}) r^{G}_{si}G_{si} + (1+ \tau_{B_{si}}) r^{K}_{si}B_{si} + (1+ \tau_{l_{si}}) w_{si}l_{si}
        \right] + {\tau_{E} E_{si}} \\
        & = C_{si} \bar{F}_{si}  \\
    \end{split}
\end{equation*}
    
    \item The optimal price level is (see appendix \ref{Ap:firm_price} for proof):
    \begin{equation}
    \begin{split}
         P_{si} =& \frac{1}{1+\tau_{si}^p}\frac{\sigma_s}{\sigma_s - 1} C_{si} \\
    \end{split}
\end{equation}

    \item Then the optimal output is:
    \begin{equation}
        \begin{split}
            Y_{si} = P_s^{\sigma_s} {Y}_s \frac{1}{P_{si} ^{\sigma_s}} = \left(\frac{P_s}{P_{si}}\right)^{\sigma_s}{Y}_s = \nu_s \frac{1}{P_{si} ^{\sigma_s}}
        \end{split}
    \end{equation}
    \item The optimal labor level is:
    \begin{equation}
        L_{si} = \frac{\nu_s}{\hat{A}_{si}P_{si} ^{\sigma_s}}  {z_{si}^l}^{-\beta_s}
    \end{equation}
\end{itemize}

\section*{Technology}
We need to find an expression for the technologies ($\hat{A}_{si}$ and $\tilde{A}_{si}$) based on observable variables. (see appendix \ref{Ap:Productiontechnology} and \ref{Ap:Emissiontechnology} for proof)

\begin{equation*}
    \hat{A}_{si} = \nu_s \dfrac{(P_{si}Y_{si})^{\frac{\sigma_s}{\sigma_s-1}}}{\hat{K}_{si}^{\beta_s} L_{si}^{1-\beta_s}}, \quad \text{where} \quad \nu_s = \frac{1}{P_s(P_sY_s)^{\frac{1}{\sigma_s-1}}}
\end{equation*}

\begin{equation*}
    \tilde{A}_{si} = \frac{E_{si}}{B_{si}}
\end{equation*}



\section*{Wedges}
\begin{itemize}
    \item The marginal nominal product of each input should be equal to the marginal cost of that for the maximizing firm (see appendix \ref{Ap:wedges} for proof).
	\begin{gather*}
    \alpha_s \beta_s  \frac{\sigma_s-1}{\sigma_s} \frac{P_{si}Y_{si}}{\hat{K}_{si}}(\frac{\hat{K}_{si}}{G_{si}})^{\frac{1}{\gamma_s}} = \frac{\partial Cost_{si}}{\partial G_{si}} \\\\
    (1-\alpha_s) \beta_s  \frac{\sigma_s-1}{\sigma_s} \frac{P_{si}Y_{si}}{\hat{K}_{si}}(\frac{\hat{K}_{si}}{B_{si}})^{\frac{1}{\gamma_s}} = \frac{\partial Cost_{si} }{\partial B_{si}}\\\\
    (1-\beta_s) \frac{\sigma_s-1}{\sigma_s} \frac{P_{si}Y_{si}}{L_{si}} = \frac{\partial Cost_{si}}{\partial L_{si}} \\\\
\end{gather*}
\end{itemize}
\section*{Estimation}
I will follow the the \cite{kmenta1967estimation},\cite{blundell2000gmm},\cite{davis2014macroeconomic} to estimate the parameters of the model. The production function is:
\begin{gather*}
    Y_{it} = \hat{A}_{it}(
        \alpha G_{it}^{\frac{\gamma-1}{\gamma}} + (1-\alpha) B_{it}^{\frac{\gamma-1}{\gamma}}
    ) ^ {\frac{\beta\gamma}{\gamma-1}} L_{it}^{1-\beta}\\
    \ln Y_{it} = \ln \hat{A}_{it} + \frac{\beta\gamma}{\gamma-1} \ln(
        \alpha G_{it}^{\frac{\gamma-1}{\gamma}} + (1-\alpha) B_{it}^{\frac{\gamma-1}{\gamma}}
    ) + (1-\beta) \ln L_{it}
\end{gather*}
as I cannot measure the production directly, I will use the total revenue in the place of the production. So, the equation becomes:
\begin{gather*}
    \ln \nu_{t}(P_{it}Y_{it})^{\frac{\sigma}{\sigma - 1}} = \ln \hat{A}_{it} + \frac{\beta\gamma}{\gamma-1} \ln(
        \alpha G_{it}^{\frac{\gamma-1}{\gamma}} + (1-\alpha) B_{it}^{\frac{\gamma-1}{\gamma}}
    ) + (1-\beta) \ln L_{it}\\
    \ln P_{it}Y_{it} = \frac{\sigma - 1}{\sigma}({\ln \hat{A}_{it} + \frac{\beta\gamma}{\gamma-1} \ln(
        \alpha G_{it}^{\frac{\gamma-1}{\gamma}} + (1-\alpha) B_{it}^{\frac{\gamma-1}{\gamma}}
    ) + (1-\beta) \ln L_{it}}) - \ln \nu_{s}
\end{gather*}
where $\nu_{s}$ is the price of the output of the sector $s$.

I take the first-order approximation of the production function around the $\gamma_s = 1$. This requires to take the limit of the third term in the production function as $\gamma_s$ approaches 1. The limit is:
\begin{gather*}
    \lim_{\gamma_s \to 1} \frac{\gamma}{\gamma-1} \ln(
        \alpha G_{it}^{\frac{\gamma-1}{\gamma}} + (1-\alpha) B_{it}^{\frac{\gamma-1}{\gamma}}
    ) = \alpha \ln G_{it}   + (1-\alpha)\ln B_{it}
\end{gather*}
and the limit of the derivative of this same term
\begin{gather*}
    \lim_{\gamma_s \to 1} \frac{\partial}{\partial \gamma_s} \left( \frac{\gamma}{\gamma-1} \ln(
        \alpha G_{it}^{\frac{\gamma-1}{\gamma}} + (1-\alpha) B_{it}^{\frac{\gamma-1}{\gamma}}
    ) \right) =  \frac{\alpha (1-\alpha)}{2}(\ln G_{it} - \ln B_{it})^2
\end{gather*}
The first-order approximation of the production function is:
\begin{equation*}
     \begin{split}
        \ln P_{it}Y_{it} \sim  & \frac{\sigma}{\sigma-1}(\ln \hat{A}_{it}) +  \frac{\sigma}{\sigma-1}\beta\alpha \ln G_{it}   + \frac{\sigma}{\sigma-1}\beta(1-\alpha)\ln B_{it} + \frac{\sigma}{\sigma-1}(1-\beta) \ln L_{it} \\
         & + \frac{\sigma}{\sigma-1}\beta \frac{\alpha (1-\alpha)(\gamma_s - 1)}{2}(\ln G_{it} - \ln B_{it})^2 - \ln \nu_{t}
    \end{split}
\end{equation*}

\begin{itemize}
    \item IV:\\
    \item Firm level:\\
    I assume that the $\hat{A}_{it} \equiv e^a e^{u_{it}}$, where $a$ is the common component of productivity and $u_{it}$ is the unpredictable component. We can rewrite the production function as:
    \begin{equation*}
        \begin{split}
           \ln P_{it}Y_{it} \sim  & \frac{\sigma}{\sigma-1}(a + u_{it}) +  \frac{\sigma}{\sigma-1}\beta\alpha \ln G_{it}   + \frac{\sigma}{\sigma-1}\beta(1-\alpha)\ln B_{it} + \frac{\sigma}{\sigma-1}(1-\beta) \ln L_{it} \\
            & + \frac{\sigma}{\sigma-1}\beta \frac{\alpha (1-\alpha)(\gamma_s - 1)}{2}(\ln G_{it} - \ln B_{it})^2 - \ln \nu_{t}
       \end{split}
    \end{equation*}
    
    To estimate this approximation, I assume that the error term can be decomposed as $u_{it} = f_i + \epsilon_{it}$ where $f_i$ varies across firm. Then I can use OLS with firm fixed effects  within each Industry and appropriately defined coefficients:
    \begin{gather*}
        \ln P_{it}Y_{it} = \beta_0^Y + \beta_G^Y \ln G_{it} + \beta_B^Y \ln B_{it} + \beta_L^Y \ln L_{it} + \beta_{GB}^Y (\ln G_{it} - \ln B_{it})^2 + \mu_i + \mu_t + \epsilon_{it}
    \end{gather*}
    
    The industry effects will be absorbed by the constant term as the estimation is done within each industry.  
    
    \item AR(1):\\
    Alternatively, I can assume that the error term is $u_{it} = z_{it} + \epsilon_{it}$ where $z_{it}$ is a predictable component that follows the AR(1) process $z_{it} = \rho z_{it-1} + \nu_{it}$. Then I will rewrite the approximation of the production function as:
    \begin{equation*}
      \begin{split}
        \ln P_{it}Y_{it} - \rho \ln P_{it-1}Y_{it-1} & = \beta_0^Y(1-\rho) + \beta_G^Y (\ln G_{it} - \rho \ln G_{it-1}) \\
        &  + \beta_B^Y (\ln B_{it} - \rho \ln B_{it-1}) + \beta_L^Y (\ln L_{it} - \rho \ln L_{it-1}) \\
        & + \beta_{GB}^Y [(\ln G_{it} -  \ln B_{it} )^2 - \rho(\ln G_{it-1} - \ln B_{it-1})^2]\\
        & + (z_{it} - \rho z_{it-1}) + (\epsilon_{it} - \rho \epsilon_{it-1})
      \end{split}
    \end{equation*}
    where $(z_{it} - \rho z_{it-1}) + (\epsilon_{it} - \rho \epsilon_{it-1}) = \nu_{it} + (\epsilon_{it} - \rho \epsilon_{it-1})$ is the zero mean error term. Then I can use GMM to estimate the four coefficients with five instruments, namely, $\ln G_{it-1}$, $\ln B_{it-1}$, $\ln L_{it-1}$, $(\ln G_{it-1} - \ln B_{it-1})^2$, and $\ln P_{it-1}Y_{it-1}$.
    \begin{equation*}
        \begin{split}
          \ln P_{it}Y_{it} & = \beta_0^Y(1-\rho) + \beta_G^Y \ln G_{it} - \rho\beta_G^Y \ln G_{it-1} \\
          &  + \beta_B^Y (\ln B_{it} - \rho \ln B_{it-1}) + \beta_L^Y (\ln L_{it} - \rho \ln L_{it-1}) \\
          & + \beta_{GB}^Y [(\ln G_{it} -  \ln B_{it} )^2 - \rho(\ln G_{it-1} - \ln B_{it-1})^2]\\
          & + \rho \ln P_{it-1}Y_{it-1}  + (z_{it} - \rho z_{it-1}) + (\epsilon_{it} - \rho \epsilon_{it-1})
        \end{split}
      \end{equation*}
\end{itemize}

Now given the estimates of the equation above, I can estimate the parameters of the production function.  First two parameters are given by the following ratios:
\begin{gather*}
    \frac{\beta_G}{\beta_B} = \frac{\alpha}{1-\alpha} \rightarrow \alpha \checkmark\\
    \frac{\beta_G}{\beta_L} = \frac{\alpha \beta}{1-\beta} \rightarrow \beta \checkmark
\end{gather*}
Now, I know the the $\alpha$ and $\beta$ parameters. I can use the following ratios to estimate the $\sigma$:
\begin{equation*}
    \begin{split}
        (1-\beta_s) \frac{\sigma_s-1}{\sigma_s} \frac{P_{si}Y_{si}}{L_{si}} = \frac{\partial Cost}{\partial L} = (1 + \tau^W_s)w_{si} \\
        \Rightarrow \frac{(1 + \tau^W_s)w_{si} L}{P_{it}Y_{it}} = \frac{W L}{P_{it}Y_{it}}= (1-\beta) \frac{\sigma-1}{\sigma} \rightarrow \sigma \checkmark
    \end{split}
\end{equation*}
Finally, Given the identified parameters, I can estimate the $\gamma$ using the following ratio:
\begin{gather*}
\frac{\beta_{GB}}{\beta_B\beta_G} = \frac{\sigma}{\sigma - 1}\frac{(\gamma-1)}{2\beta} \rightarrow \gamma \checkmark
\end{gather*}

Given the estimated parameters of the production function, I set the production productivity to match the total revenue in the sector. Then I can estimate the emission inefficiency to match the emission intensity of the sector. The estimated parameters are given in the Table~\ref{tab:estimate_parameters}.

\begin{table}[http]
    \centering
    \caption{\quad Parameters to be estimated} \label{tab:estimate_parameters}
    \vspace*{2mm}
    \begin{tabular}{@{}cc@{}}
    \toprule
    \textbf{Parameter}  & \textbf{Source/Moment}\\
    \midrule 
    \multicolumn{2}{c}{Panel A: Estimation} \\
    \midrule
    $\alpha_s$ & ${\beta_G}/{\beta_B}$ \\
    $\beta_s$ & ${\beta_G}/{\beta_L}$\\ 
    $\sigma_s$ & $WL/PY$\\
    $\gamma_s$ & ${\beta_{GB}}/{\beta_B\beta_G}$\\
    $Mean(\log(\hat{A}_{si})) $ & $Mean({L}_{si})$\\
    $Sd(\log({\hat{A}}))$ & $Sd({L}_{si})$\\
    $Mean(\log(\tilde{A}_{si})) $ & $Mean({E/PY})$\\
    $Sd({\log(\tilde{A})})$ & $Sd({E/PY})$\\
    $Corr(\log(\hat{A}),\log(\tilde{A}))$ & $Corr(PY,E/PY)$\\
    \midrule
    \multicolumn{2}{c}{Panel B: Calibration} \\
    \midrule
    $r$ & $5\%$  \\
    $w$ & $500$ TSEK  \\
    \midrule
    \multicolumn{2}{c}{Panel C: Additional Moments} \\
    \midrule
    $ \left(
            \frac{\alpha}{1-\alpha} \frac{r^B + {\tau_E}\tilde{A}}{r^G}
        \right) ^{\frac{1}{\gamma}}$ & $z_k= G/B$ \\\\
    $\frac{1-\beta}{\beta}\frac{1}{\alpha}\left(
        \alpha + (1-\alpha) {{z_k}}^{1-\gamma}
    \right)^{\frac{1}{1-\gamma}} \frac{r^G}{W}$ & $z_l = L/K$ \\\\
    $\frac{1}{\gamma} \frac{\tilde{A}}{r^B + \tau_E\tilde{A}} z_k$ & ${\partial z_k/\partial \tau_E }$ \\\\
    $\frac{\tilde{A}/\gamma}{r^B + \tau_E\tilde{A}}\frac{1}{\frac{\alpha}{1-\alpha}{{z_k}}^{\gamma-1} + 1}{z_l}$ &   $\partial z_l/\partial \tau_E$\\\\
    - & $\Delta(\frac{E}{PY})/\Delta(\frac{C}{PY}) $ \\
    \bottomrule
\end{tabular}
\end{table}





% Under either of these assumptions, I can estimate the parameters of the production function and then use the same approach to estimate the emission function\footnote{ The estimation model is:
% \begin{gather*}
%     \ln E_{it} = \beta_0^E + \beta_G^E \ln G_{it} + \beta_B^E \ln B_{it} + \beta_L^E \ln L_{it} + \beta_{GB}^E (\ln G_{it} - \ln B_{it})^2 + \epsilon_{it}
% \end{gather*}}. The estimates of the parameters are given in the Table~\ref{tab:estimate_parameters}.

% \begin{table}[http]
%     \centering
%     \caption{\quad Parameters estimates of production and emission functions} \label{tab:estimate_parameters}
%     \begin{tabular}{cc}
%         \hline
%         \textbf{Parameter} & \textbf{Value} \\
%         \hline
%         $\beta_{s}$ & $1-\beta_L^Y$ \vspace{5pt}
%         \\
%         $\alpha_{s}$ & $1-\dfrac{\beta_B^Y}{1-\beta_L^Y}$ \vspace{5pt}
%         \\
%         $\gamma_{s}$ & $1+\dfrac{2\beta_{GB}^Y(1-\beta_L^Y)}{\beta_G^Y\beta_B^Y}$ \vspace{5pt}
%         \\
%         \hline
%     \end{tabular}
%     \hspace{1cm}
%     \begin{tabular}{cc}
%       \hline
%       \textbf{Parameter} & \textbf{Value} \\
%       \hline
%       $\theta_{s}$ & $1-\beta_L^E$ \vspace{5pt}
%       \\
%       $\mu_{s}$ & $1-\dfrac{\beta_B^E}{1-\beta_L^E}$ \vspace{5pt}
%       \\
%       $\eta_{s}$ & $1+\dfrac{2\beta_{GB}^E(1-\beta_L^E)}{\beta_G^E\beta_B^E}$ \vspace{5pt}
%       \\
%       \hline
%   \end{tabular}
% \end{table}



\clearpage


\section*{Additional Moments}
\begin{enumerate}
    \item 
    \begin{equation*}
        \begin{split}
            \ln z_k =& \frac{1}{\gamma} (
            \ln \frac{\alpha}{1-\alpha} + \ln ({r^B + \tau_E\tilde{A}}) - \ln {\tilde{A}} )\\
            \Rightarrow & \frac{\partial z_k/\partial \tau_E }{z_k} = \frac{1}{\gamma} \frac{\tilde{A}}{r^B + \tau_E\tilde{A}}\\
        \end{split}
    \end{equation*}
    \item 
    \begin{equation*}
        \begin{split}
            \ln z_l =& \ln \frac{1-\beta}{\beta} - \ln {\alpha} + \frac{1}{1-\gamma} \ln (\alpha + (1-\alpha) {{z_k}}^{1-\gamma}) + \ln r^G - \ln W\\
            \Rightarrow & \frac{\partial z_l/\partial \tau_E }{z_l} =  \frac{(1-\alpha)z_k^{-\gamma}}{\alpha + (1-\alpha) {{z_k}}^{1-\gamma}}\frac{\partial z_k}{\partial \tau_E} = \frac{\tilde{A}/\gamma}{r^B + \tau_E\tilde{A}}\frac{1}{\frac{\alpha}{1-\alpha}{{z_k}}^{\gamma-1} + 1}
        \end{split}
    \end{equation*}
    
    \item 
    \begin{equation*}
        \begin{split}
            & \frac{E}{\nu_{s}(PY)^{\frac{\sigma_s}{\sigma_s - 1}}}  = \frac{\tilde{A}}{\hat{A}}\left(
                \alpha_s {{z^k}}^{\gamma_s - 1} + (1-\alpha_s)
            \right) ^ {\frac{\gamma_s}{1-\gamma_s}} {{z_l}}^{1 - \beta} , \quad \epsilon \equiv  \frac{E}{PY}\\
            \Rightarrow & \ln \epsilon =  \ln \frac{\tilde{A}}{\hat{A}} + \frac{\gamma_s}{1-\gamma_s} \ln \left(
                \alpha_s {{z^k}}^{\gamma_s - 1} + (1-\alpha_s)
            \right) + (1-\beta) \ln {{z_l}} + \frac{1}{\sigma -1} \ln R + \ln \nu_{s} \\
            \Rightarrow &\ln \epsilon \sim \ln \frac{\tilde{A}}{\hat{A}} + \alpha \ln z_k + \frac{\alpha(1-\alpha)(\gamma-1)}{2} (\ln z_k)^2 + (1-\beta) \ln {{z_l}} + \frac{1}{\sigma -1} \ln R + \ln \nu_{s} \\
            \Rightarrow& \ln \epsilon = \mu_i + \mu_t + \beta_{z_k}^{\epsilon} \ln z_k + \beta_{z_k^2}^{\epsilon} (\ln z_k)^2 + \beta_{z_l}^{\epsilon} \ln z_l + \beta_{R}^{\epsilon} \ln R  \\\\
            &     \qquad\qquad\qquad        \beta_{z_k}^{\epsilon} = \alpha \rightarrow \alpha \checkmark\\
            &     \qquad\qquad\qquad        \beta_{z_k^2}^{\epsilon} = \frac{\alpha(1-\alpha)(\gamma-1)}{2}\rightarrow \gamma \checkmark \\
            &     \qquad\qquad\qquad        \beta_{z_l}^{\epsilon} = 1-\beta \rightarrow \beta\checkmark\\
            &     \qquad\qquad\qquad        \beta_{R}^{\epsilon} = \frac{1}{\sigma -1} \rightarrow \sigma \checkmark
            \\\\\\
            \Rightarrow & \frac{\partial \epsilon/\partial \tau_E}{\epsilon} =  -\frac{\gamma_s}{\alpha_s {{z^k}}^{\gamma_s - 1} + (1-\alpha_s)} {\alpha_s z_k^{\gamma_s}} \frac{\partial z_k}{\partial \tau_E} + \frac{1-\beta}{z_l} \frac{\partial z_l}{\partial \tau_E} + \frac{1}{(\sigma -1)R} \frac{\partial R}{\partial \tau_E}
        \end{split}
    \end{equation*}
\end{enumerate}

\clearpage

\section{Definitions and Proofs} 
\subsection{Optimal level for given output} \label{Ap:optimal level for given output}
\begin{equation*}
    \bar{Y}_{si} = \hat{A}_{si} \hat{K^*}_{si}^{\beta_s} {L^*}_{si}^{1-\beta_s} = \hat{A}_{si} \hat{K^*}_{si} ({\frac{L^*}{\hat{K^*}}})_{si}^{(1-\beta_s)}= \hat{A}_{si} \hat{K^*}_{si} {z_{si}^l}^{(1-\beta_s)}
\end{equation*}
we can rewrite the optimal capital level as:
\begin{equation*}
    \begin{split}
        \hat{K}_{si} &= B_{si}(\alpha_s {z^k_{si}}^{\frac{\gamma_s -1}{\gamma_s}} + (1-\alpha_s))^{\frac{\gamma_s}{\gamma_s - 1}}\\
        & = G_{si}(\alpha_s  + (1-\alpha_s){z^k_{si}}^{(1-\gamma_s )})^{\frac{\gamma_s}{\gamma_s - 1}}
    \end{split}
\end{equation*}
and then drive the optimal level of each type of capital:
\begin{equation*}
    G_{si}^* = \frac{\bar{Y}_{si}}{\hat{A}_{si}} \left(
        \alpha_s  + (1-\alpha_s){z^k_{si}}^{(1-\gamma_s )}
    \right)^{\frac{\gamma_s}{1-\gamma_s}} {z_{si}^l}^{(1-\beta_s)}
\end{equation*}
\begin{equation*}
    B_{si}^* = \frac{\bar{Y}_{si}}{\hat{A}_{si}} \left(\alpha_s {z^k_{si}}^{\frac{\gamma_s -1}{\gamma_s}} + (1-\alpha_s)
    \right)^{\frac{\gamma_s}{1-\gamma_s}} {z_{si}^l}^{(\beta_s-1)}
\end{equation*}
\begin{equation*}
    L_{si}^* = \frac{\bar{Y}_{si}}{\hat{A}_{si}}  {z_{si}^l}^{-\beta_s}
\end{equation*}

\subsection{Emission optimal level} \label{Ap:emission optimal level}
First we need to convert the production capital ($\hat{K}_{si}$) to the emission capital ($\tilde{K}_{si}$):
\begin{equation*}
    \begin{split}
        \tilde{K} = (
        \mu_s G_{si}^{\frac{\eta_s-1}{\eta_s}} + (1-\mu_s) B_{si}^{\frac{\eta_s-1}{\eta_s}}
    ) ^ {\frac{\eta_s}{\eta_s-1}} = G_{si}(\mu_s  + (1-\mu_s){z^k_{si}}^{(1-\eta_s )})^ {\frac{\eta_s}{\eta_s-1}}\\
    \hat{K} = (
        \alpha_s G_{si}^{\frac{\gamma_s-1}{\gamma_s}} + (1-\alpha_s) B_{si}^{\frac{\gamma_s-1}{\gamma_s}}
    ) ^ {\frac{\gamma_s}{\gamma_s-1}} = G_{si}(\alpha_s  + (1-\alpha_s){z^k_{si}}^{(1-\gamma_s )})^ {\frac{\gamma_s}{\gamma_s-1}}
    \end{split}
\end{equation*}
\begin{equation*}
    \Rightarrow \phi_{si} \equiv \frac{\tilde{K}}{\hat{K}} = \frac{(\mu_s  + (1-\mu_s){z^k_{si}}^{(1-\eta_s )})^ {\frac{\eta_s}{\eta_s-1}}}{(\alpha_s  + (1-\alpha_s){z^k_{si}}^{(1-\gamma_s )}) ^{\frac{\gamma_s}{\gamma_s-1}}} = \frac{(\mu_s {z^k_{si}}^{(\eta_s -1)}  + (1-\mu_s))^ {\frac{\eta_s}{\eta_s-1}}}{(\alpha_s {z^k_{si}}^{(\gamma_s -1)}  + (1-\alpha_s))^{\frac{\gamma_s}{\gamma_s-1}}}
\end{equation*}
As we can see the $\phi_{si}$ is a function of $z^k_{si}$, $\mu_s$, $\alpha_s$ and $\gamma_s$. Now we can calculate the optimal emission level ($E_{si}$) as follows:
\begin{equation*}
    \begin{split}
        E_{si} &= \tilde{A}_{si}\tilde{K}_{si}^{\theta_s} L_{si}^{1-\theta_s} = \tilde{A}_{si} \tilde{K}_{si} ({\frac{L_{si}}{\tilde{K}_{si}}})^{1-\theta_s}\\
    &= \tilde{A}_{si} (\phi_{si} \hat{K}_{si}) ({\frac{L_{si}}{\phi_{si} \hat{K}_{si}}})^{1-\theta_s}\\
    &= {\tilde{A}_{si}}{\phi_{si}}^{\theta_s} \hat{K}_{si} {z^{l}_{si}}^{1-\theta_s} \\
   & = {\tilde{A}_{si}}{\phi_{si}}^{\theta_s}{z^{l}_{si}}^{\beta_s-\theta_s}\hat{K}_{si}{z^{l}_{si}}^{1-\beta_s}\\
   & = \frac{\tilde{A}_{si}}{\hat{A}_{si}}(\frac{\phi_{si}}{z^{l}_{si}})^{\theta_s} {z^{l}_{si}}^{\beta_s} \bar{Y}_{si}
    \end{split}
\end{equation*}


\subsection{Cost Minimization function} \label{Ap:cost minimization function}
\begin{equation*}
	C(\bar{F}_{si}) = C_{si} \bar{F}_{si} 
	\end{equation*}
	\begin{equation*}
		\begin{split}
			\text{where} \quad C_{si}  &  = (1+ \tau_{G_{si}}) r^{G}_{si} \frac{1}{\hat{A}_{si}}\left(
				\alpha_s  + (1-\alpha_s){z^k_{si}}^{-\frac{\gamma_s-1}{\gamma_s}}
				\right)^{\frac{\gamma_s}{1-\gamma_s}} {z_{si}^l}^{(\beta_s-1)}\\
			& + (1+ \tau_{B_{si}})r^{B}_{si}\frac{1}{\hat{A}_{si}} \left(\alpha_s {z^k_{si}}^{\frac{\gamma_s -1}{\gamma_s}} + (1-\alpha_s)
			\right)^{\frac{\gamma_s}{1-\gamma_s}} {z_{si}^l}^{(\beta_s-1)}\\
			& + (1+ \tau_{l_{si}}) w_{si} \frac{1}{\hat{A}_{si}}  {z_{si}^l}^{\beta_s}\\
			 & + \tau_{E} \frac{\tilde{A}_{si}}{\hat{A}_{si}}(\frac{\phi_{si}}{z^{l}_{si}})^{\theta_s} {z^{l}_{si}}^{\beta_s}
		\end{split}
	\end{equation*}

\subsection{Sector Price} \label{Ap:sector price}
 We need to solve the sector price $P_s$ as function of firm price $P_{si}$, where $P_s$ is defined as the price of acquiring a unit of the sector benefit:
    \begin{equation*}
    \begin{split}
        \min_{F_{si}} & \quad \left\{
    \sum_{i} P_{si} F_{si}	
    \right\}\\
    \text{s.t.} & \quad \left(\sum_{i} F_{si}^{\frac{\sigma_s-1}{\sigma_s}}\right)^{\frac{\sigma_s}{\sigma_s-1}} = \bar{F}_s
    \end{split}
\end{equation*}
\begin{equation*}
    F.O.C \Rightarrow P_{si}^{\sigma_s} F_{si} = P_s^{\sigma_s} {F}_s
\end{equation*}

\subsection{Firm Price} \label{Ap:firm_price}
We need to solve the firm price $P_{si}$ as function of sector price $P_s$, where $P_{si}$ is defined as the price of acquiring a unit of the firm benefit:
    \begin{gather*}
    \max_{P_{si}} \quad\pi_{si} =  (1+\tau_{si}^p)P_{si}Y_{si} - C_{si} {Y}_{si} \\
    \max  \quad (1+\tau_{si}^p)P_{si}( \frac{P_s}{ P_{si}})^{\sigma_s}{Y}_s - C_{si} (\frac{P_s}{ P_{si}})^{\sigma_s}{Y}_s \\
    \max \quad  (1+\tau_{si}^p)P_{si}^{1-\sigma_s}P_s^{\sigma_s}{Y}_s - C_{si} P_{si}^{-\sigma_s}P_s^{\sigma_s}{Y}_s \\
    \max \quad P_s^{\sigma_s}{Y}_s \left(
        (1+\tau_{si}^p)P_{si}^{1-\sigma_s} - C_{si} P_{si}^{-\sigma_s} 
    \right)\\
    \max \quad  (1+\tau_{si}^p)P_{si}^{1-\sigma_s} - C_{si} P_{si}^{-\sigma_s} \\
    \Rightarrow 0 = P_{si}^{-\sigma_s - 1} \left[
        (1+\tau_{si}^p)(1-\sigma_s)P_{si} + \sigma_s C_{si} 
    \right]
    \\
    \Rightarrow P_{si} = \frac{1}{1+\tau_{si}^p}\frac{\sigma_s}{\sigma_s - 1} C_{si} 
\end{gather*}

\subsection{Production Technology}
\label{Ap:Productiontechnology}
\begin{gather*}
    P_{si}^{\sigma_s} = P_s^{\sigma_s} Y_s Y_{si}^{-1} 
    \Rightarrow (P_{si}Y_{si})^{\sigma_s} = P_s^{\sigma_s} Y_s Y_{si}^{\sigma_s-1}
    \Rightarrow (P_{si}Y_{si})^{\sigma_s} = (P_sY_s)(P_sY_{si})^{\sigma_s-1}\\
    \Rightarrow (P_{si}Y_{si})^{\frac{\sigma_s}{\sigma_s-1}} = (P_sY_s)^{\frac{1}{\sigma_s-1}}P_sY_{si}\\
    \Rightarrow Y_{si} = \dfrac{(P_{si}Y_{si})^{\frac{\sigma_s}{\sigma_s-1}}}{P_s(P_sY_s)^{\frac{1}{\sigma_s-1}}}\\
    \Rightarrow \hat{A}_{si} = \frac{1}{P_s(P_sY_s)^{\frac{1}{\sigma_s-1}}}\dfrac{(P_{si}Y_{si})^{\frac{\sigma_s}{\sigma_s-1}}}{\hat{K}_{si}^{\beta_s} L_{si}^{1-\beta_s}} = \nu_s \dfrac{(P_{si}Y_{si})^{\frac{\sigma_s}{\sigma_s-1}}}{\hat{K}_{si}^{\beta_s} L_{si}^{1-\beta_s}}
\end{gather*}
\subsection{Emission Technology}
\label{Ap:Emissiontechnology}
\begin{gather*}
    E_{si} = \tilde{A}_{si}\tilde{K}_{si}^{\theta_s} L_{si}^{1-\theta_s} \Rightarrow \tilde{A}_{si} = \dfrac{E_{si}}{\tilde{K}_{si}^{\theta_s} L_{si}^{1-\theta_s}}\\
\end{gather*}



\subsection{Reallocation} \label{Ap:reallocation}


It is the social planner's problem to reallocate resources, therefore there is no need to price the inputs. The social planner's problem is to maximize the total real output subject to the resource constraint. Firms in the economy will produce the optimal level of output given the zero cost of inputs. The social planner's problem is:
\input{model_elements/social planner's problem}

Now I will write the lagrangian for the social planner's problem. The social planner's problem is to maximize the total output in the economy subject to the resource constraint. The lagrangian is:
\input{model_elements/social planner's problem - largranian}
where $\lambda_L$, $\lambda_E$ are the lagrange multipliers for the labor and emission constraints, respectively that are normalized by the capital lagrange multiplier. This results in the fact that the relative marginal cost of labor is $\lambda_L$ and the relative marginal cost of acquiring one unit of capital is 1. Now I can use the results from the firm's problem and the optimal input ratios to derive the optimal output and emissions for the social planner. The optimal input ratios are:
\begin{gather}\label{eq:reallocation_firm_input_ratio}
    z_{s}^k = (\frac{\alpha_s}{1-\alpha_s})^{\gamma_s} \\
    z_{s}^l = \frac{1-\beta_s}{\beta_s} \frac{1}{1-\alpha_s} (\alpha_s {z^k_{s}}^{(\gamma_s -1)} + (1-\alpha_s))^{\frac{1}{1-\gamma_s}} \lambda_L
  \end{gather}
Then I can rewrite the optimal output and emissions for firm under the social planner's problem as:
\input{model_elements/social planner's problem - firm output}
and rewrite the lagrangian as:
\begin{gather*}
    \mathcal{L}_s = \left(
        \sum_i \hat{A}_{si} {z_{s}^l}^{-\beta}\hat{L}_{si}^{\frac{\sigma-1}{\sigma}}
    \right)^{\frac{\sigma}{\sigma-1}} + \lambda_L(L_s - \sum_i \hat{L}_{si}) + \lambda_E(E_s - \sum_i \tilde{A}_{si}(\frac{\phi_{s}}{z^{l}_{s}})^{\theta_s} \hat{L}_{si})\\
    \frac{\partial \mathcal{L}_s}{\partial \hat{L}_{si}} = \left(
        \sum_i \hat{A}_{sk} {z_{s}^l}^{-\beta}\hat{L}_{sk}^{\frac{\sigma-1}{\sigma}}
    \right)^{\frac{1}{\sigma-1}}(\hat{A}_{si}{z_{s}^l}^{-\beta})^{\frac{\sigma-1}{\sigma}}\hat{L}_{si}^{-\frac{1}{\sigma}} - \lambda_L - \lambda_E \tilde{A}_{si}(\frac{\phi_{s}}{z^{l}_{s}})^{\theta_s} = 0\\
    \frac{\partial \mathcal{L}_s}{\partial \hat{L}_{sj}} = \left(
        \sum_i \hat{A}_{sk} {z_{s}^l}^{-\beta}\hat{L}_{sk}^{\frac{\sigma-1}{\sigma}}
    \right)^{\frac{1}{\sigma-1}}(\hat{A}_{sj}{z_{s}^l}^{-\beta})^{\frac{\sigma-1}{\sigma}}\hat{L}_{sj}^{-\frac{1}{\sigma}} - \lambda_L - \lambda_E \tilde{A}_{sj}(\frac{\phi_{s}}{z^{l}_{s}})^{\theta_s} = 0
\end{gather*}

There are two possible cases for the optimal allocation of resources. In the first case, the social planner only cares about maximizing total output and does not consider emissions, resulting in $\lambda_E = 0$. In this scenario, the social planner will allocate all available resources to the firm with the highest productivity:
\begin{gather*}
    \lambda_L = \left(
        \sum_i \hat{A}_{sk} {z_{s}^l}^{-\beta}\hat{L}_{sk}^{\frac{\sigma-1}{\sigma}}
    \right)^{\frac{1}{\sigma-1}}(\hat{A}_{si}{z_{s}^l}^{-\beta})^{\frac{\sigma-1}{\sigma}}\hat{L}_{si}^{-\frac{1}{\sigma}} = \left(
        \sum_i \hat{A}_{sk} {z_{s}^l}^{-\beta}\hat{L}_{sk}^{\frac{\sigma-1}{\sigma}}
    \right)^{\frac{1}{\sigma-1}}(\hat{A}_{sj}{z_{s}^l}^{-\beta})^{\frac{\sigma-1}{\sigma}}\hat{L}_{sj}^{-\frac{1}{\sigma}}\\
    \Rightarrow (\frac{\hat{A}_{si}^{\sigma-1}}{\hat{L}_{si}})^{\frac{1}{\sigma}} = (\frac{\hat{A}_{sj}^{\sigma-1}}{\hat{L}_{sj}})^{\frac{1}{\sigma}} \Rightarrow \frac{\hat{L}_{si}}{\hat{L}_{sj}} = \left(
        \frac{\hat{A}_{si}}{\hat{A}_{sj}}
    \right)^{\sigma-1}
\end{gather*}
 
In the second case, the social planner cares about the emissions, and the social planner will allocate all available resources to the firm with the lowest emissions:
\begin{gather*}
    \lambda_E = \dfrac{\left(
        \sum_i \hat{A}_{sk} {z_{s}^l}^{-\beta}\hat{L}_{sk}^{\frac{\sigma-1}{\sigma}}
    \right)^{\frac{1}{\sigma-1}}(\hat{A}_{si}{z_{s}^l}^{-\beta})^{\frac{\sigma-1}{\sigma}}\hat{L}_{si}^{-\frac{1}{\sigma}}}{
        \tilde{A}_{si}(\frac{\phi_{s}}{z^{l}_{s}})^{\theta_s}
    } = \dfrac{\left(
        \sum_i \hat{A}_{sk} {z_{s}^l}^{-\beta}\hat{L}_{sk}^{\frac{\sigma-1}{\sigma}}
    \right)^{\frac{1}{\sigma-1}}(\hat{A}_{sj}{z_{s}^l}^{-\beta})^{\frac{\sigma-1}{\sigma}}\hat{L}_{sj}^{-\frac{1}{\sigma}}}{
        \tilde{A}_{sj}(\frac{\phi_{s}}{z^{l}_{s}})^{\theta_s}
    }\\
    \Rightarrow \frac{\hat{A}_{si}^{\frac{\sigma-1}{\sigma}}}{\tilde{A}_{si}\tilde{L}_{si}^{\frac{1}{\sigma}}} = \frac{\hat{A}_{sj}^{\frac{\sigma-1}{\sigma}}}{\tilde{A}_{sj}\tilde{L}_{sj}^{\frac{1}{\sigma}}} \Rightarrow \frac{\tilde{L}_{si}}{\tilde{L}_{sj}} = \dfrac{\hat{A}_{si}^{\sigma-1}/\tilde{A}_{si}^{\sigma}}{\hat{A}_{sj}^{\sigma-1}/\tilde{A}_{sj}^{\sigma}}
\end{gather*}
as we know the optimal ratio, we can make sure that the allocation satisfies the budget constraint. The optimal allocation of resources is:
\begin{gather*}\label{eq:allocation_social_planner_L}
    \hat{L}_{si} = \dfrac{\hat{A}_{si}^{\sigma -1}}{\sum_j \hat{A}_{sj}^{\sigma -1}}L_s\\ 
    \tilde{L}_{si} = \dfrac{\hat{A}_{si}^{\sigma -1}/\tilde{A}_{si}^{\sigma}}{\sum_j \hat{A}_{sj}^{\sigma -1}/ \tilde{A}_{sj}^{\sigma}}L_s
\end{gather*}
and we know that the social planner will decide about the green and brown capital allocation in the sector. The optimal allocation of green and brown capital is:
\begin{gather*}
    B_s = \dfrac{1}{1 + z_s^k} K_s \\ 
    G_s = \dfrac{z_s^k}{1 + z_s^k} K_s
\end{gather*}
\begin{gather*}\label{eq:allocation_social_planner_G}
    \hat{G}_{si} = \dfrac{\hat{A}_{si}^{\sigma -1}}{\sum_j \hat{A}_{sj}^{\sigma -1}}G_s\\ 
    \tilde{G}_{si} = \dfrac{\hat{A}_{si}^{\sigma -1}/\tilde{A}_{si}^{\sigma}}{\sum_j \hat{A}_{sj}^{\sigma -1}/ \tilde{A}_{sj}^{\sigma}}G_s
\end{gather*}
\begin{gather*} \label{eq:allocation_social_planner_B}
    \hat{B}_{si} = \dfrac{\hat{A}_{si}^{\sigma -1}}{\sum_j \hat{A}_{sj}^{\sigma -1}}B_s\\ 
    \tilde{B}_{si} = \dfrac{\hat{A}_{si}^{\sigma -1}/\tilde{A}_{si}^{\sigma}}{\sum_j \hat{A}_{sj}^{\sigma -1}/ \tilde{A}_{sj}^{\sigma}}B_s
\end{gather*}


\subsection{Wedges} \label{Ap:wedges}
From the \ref{Ap:sector price} we know that $Y_{si} = (\frac{P_s}{ P_{si}})^{\sigma_s}{Y}_s $ and then we can put it in the nominal output for firms:
\begin{gather*}
    P_{si}Y_{si} = P_s{Y}_s^{\frac{1}{\sigma_s}}Y_{si}^{\frac{\sigma_s-1}{\sigma_s}}
\end{gather*}
\begin{equation*}
    \begin{split}
        \frac{\partial P_{si}Y_{si} }{\partial G_{si}}  & = 
        \frac{\sigma_s-1}{\sigma_s}P_s{Y}_s^{\frac{1}{\sigma_s}}Y_{si}^{\frac{-1}{\sigma_s}}\frac{\partial Y_{si}}{\partial G_{si}}\\
        & = \frac{\sigma_s-1}{\sigma_s}P_s{Y}_s^{\frac{1}{\sigma_s}}Y_{si}^{\frac{-1}{\sigma_s}} \beta_s \frac{Y_{si}}{\hat{K}_{si}}\frac{\partial \hat{K}_{si}}{\partial G_{si}}\\
        & = \frac{\sigma_s-1}{\sigma_s}P_s{Y}_s^{\frac{1}{\sigma_s}}Y_{si}^{\frac{-1}{\sigma_s}} \beta_s \frac{Y_{si}}{\hat{K}_{si}}\alpha_s G_{si}^{\frac{-1}{\gamma_s}} \hat{K}_{si}^{\frac{1}{\gamma_s}}\\
    \end{split}
\end{equation*}
From the \ref{Ap:sector price} we know that $P_{si} = P_s^{\sigma_s}Y_s^{\frac{1}{\sigma_s}}Y_{si}^{\frac{-1}{\sigma_s}}$ and then we can use it and then
\begin{equation*}
    \begin{split}
        \frac{\partial P_{si}Y_{si} }{\partial G_{si}} & = \frac{\sigma_s-1}{\sigma_s}P_{si} \beta_s \frac{Y_{si}}{\hat{K}_{si}}\alpha_s G_{si}^{\frac{-1}{\gamma_s}} \hat{K}_{si}^{\frac{1}{\gamma_s}}\\
        & = \alpha_s \beta_s  \frac{\sigma_s-1}{\sigma_s} \frac{P_{si}Y_{si}}{\hat{K}_{si}}(\frac{\hat{K}_{si}}{G_{si}})^{\frac{1}{\gamma_s}}
    \end{split}
\end{equation*}
And then we can derive the marginal benefit of the brown capital:

\begin{equation*}
    \begin{split}
        \frac{\partial P_{si}Y_{si} }{\partial B_{si}}  & = 
        \frac{\sigma_s-1}{\sigma_s}P_s{Y}_s^{\frac{1}{\sigma_s}}Y_{si}^{\frac{-1}{\sigma_s}}\frac{\partial Y_{si}}{\partial B_{si}}\\
        & = \frac{\sigma_s-1}{\sigma_s}P_s{Y}_s^{\frac{1}{\sigma_s}}Y_{si}^{\frac{-1}{\sigma_s}} \beta_s \frac{Y_{si}}{\hat{K}_{si}}\frac{\partial \hat{K}_{si}}{\partial B_{si}}\\
        & = \frac{\sigma_s-1}{\sigma_s}P_s{Y}_s^{\frac{1}{\sigma_s}}Y_{si}^{\frac{-1}{\sigma_s}} \beta_s \frac{Y_{si}}{\hat{K}_{si}}(1-\alpha_s) B_{si}^{\frac{-1}{\gamma_s}} \hat{K}_{si}^{\frac{1}{\gamma_s}}\\
        & = \frac{\sigma_s-1}{\sigma_s}P_{si} \beta_s \frac{Y_{si}}{\hat{K}_{si}}(1-\alpha_s) B_{si}^{\frac{-1}{\gamma_s}} \hat{K}_{si}^{\frac{1}{\gamma_s}}\\
        & = (1-\alpha_s) \beta_s  \frac{\sigma_s-1}{\sigma_s} \frac{P_{si}Y_{si}}{\hat{K}_{si}}(\frac{\hat{K}_{si}}{B_{si}})^{\frac{1}{\gamma_s}}
    \end{split}
\end{equation*}
And finally, we can derive the marginal benefit of labor:

\begin{equation*}
    \begin{split}
        \frac{\partial P_{si}Y_{si} }{\partial L_{si}}  & = 
        \frac{\sigma_s-1}{\sigma_s}P_s{Y}_s^{\frac{1}{\sigma_s}}Y_{si}^{\frac{-1}{\sigma_s}}\frac{\partial Y_{si}}{\partial L_{si}}\\
        & = \frac{\sigma_s-1}{\sigma_s}P_s{Y}_s^{\frac{1}{\sigma_s}}Y_{si}^{\frac{-1}{\sigma_s}} (1-\beta_s) \frac{Y_{si}}{L_{si}}\\
        &= (1-\beta_s) \frac{\sigma_s-1}{\sigma_s} \frac{P_{si}Y_{si}}{L_{si}}
    \end{split}
\end{equation*}
\subsection*{}
\bibliographystyle{aea}
\bibliography{literature}
\end{document}